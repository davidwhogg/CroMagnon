%% This file is part of the CroMagnon project
%% Copyright 2015 David W. Hogg

% To-Do
% -----
% - Clean up all mentions of ``DWH'' in the code.

\documentclass[12pt]{article}

% useless formatting
\linespread{1.08333} % 10/13 spacing
\setlength{\parindent}{2\baselineskip}\addtolength{\parindent}{-1.25ex}
\setlength{\parskip}{0ex}

% math definitions
\newcommand{\unitvec}[1]{\hat{#1}}
\newcommand{\ehat}{\unitvec{e}}
\newcommand{\xhat}{\unitvec{x}}
\newcommand{\yhat}{\unitvec{y}}
\newcommand{\zhat}{\unitvec{z}}
\newcommand{\normal}{N}
\newcommand{\given}{\,|\,}
\newcommand{\like}{L}
\newcommand{\dd}{\mathrm{d}}
\newcommand{\inverse}{^{-1}}
\newcommand{\transpose}{^{\mathsf{T}}}

\begin{document}

\section*{Inferring a 3-d Gaussian variance tensor from 2-d projections at unknown angles}
\noindent
David W. Hogg (NYU) (SCDA) (MPIA)

\bigskip

\section{Problem statement}

In a three-dimensional (3-d) space there exists a 3-d multivariate
Gaussian-shaped density blob $\rho(x)$
\begin{eqnarray}
  \rho(x) &=& a\,\normal(x\given 0, V)
  \quad ,
\end{eqnarray}
where $a$ is an amplitude and $\normal(x\given\mu, V)$ is a Gaussian
pdf for $x$ with mean $\mu$ and variance tensor $V$.
Without loss of generality we have presumed zero mean $\mu$, and a
variance tensor $V$ with principal axes aligned with the coordinate
axes $\ehat_1$, $\ehat_2$, and $\ehat_3$ directions:
\begin{eqnarray}
  V &=& \sum_{d=1}^3 \sigma^2_d \, \ehat_d\cdot\ehat_d\transpose
  \quad ,
\end{eqnarray}
where $V$ is a symmetric $3\times3$ tensor, the vectors are column
orthonormal unit vectors (so the vector products
$\ehat_d\cdot\ehat_d\transpose$ are outer products), the $V_d$ are
scalar eigenvalues (principal variances).
Still without loss of generality we can assume that the corresponding
eigenvalues of the tensor are ordered
\begin{eqnarray}
  \sigma^2_1 \geq \sigma^2_2 \geq \sigma^2_3 > 0
  \quad .
\end{eqnarray}

We don't get to observe this Gaussian blob or its variance tensor
directly.
Instead, we get $N$ two-dimensional (2-d) images $y_n$, each of which
contains a noisy 2-d projection of the 3-d Gaussian blob,
projected along an unknown three-vector direction $\zhat_n$.
In detail, each image $y_n$ is a square array of $M$ pixels at integer
coordinates $\xi_m$ in the projection plane, and each image pixel $m$
contains an evaluation of a shifted (that is, non-zero mean),
arbitrarily rotated (in 2-d), 2-d Gaussian blob plus some additive iid
Gaussian noise.
None of the projection directions $\zhat_n$, the 2-d rotations, and
the 2-d shifts are known; only the noisy images are given.

The variance tensor of the 2-d Gaussian blob appearing in image $y_n$
is given by the projection of the 3-d variance tensor:
\begin{eqnarray}
  V_n &=& P_n\transpose\cdot V\cdot P_n
  \quad ,
\end{eqnarray}
where $P_n$ is a $3\times 2$ rectangular projection matrix.
The projection matrices $P_n$ have three degrees of freedom or three
parameters (in some sense), which can be thought of as the Euler
angles, which are two angles on the sphere to define the direction of
$\zhat_n$ and one more angle to define the direction of the x-axis of
the image in the projection plane.
DWH: INSERT MATH HERE.

DWH: SHOW EXAMPLE DATA HERE.

The question is:
Given these data, \emph{can we infer the 3-d variance tensor $V$?}
And if we can, how accurately, and what does it depend on?
We are going to attempt the inference by maximizing a marginalized
likelihood, marginalizing out all $5N$ unknown projection, rotation,
and shift parameters.

There is another similar problem we could pose in which each datum
$y_n$ is a \emph{sampling} from a two-dimensional projection of the
Gaussian blob, possibly also with a uniform background (dc) component.
That's beyond our current scope!

\section{Marginalized likelihood and the sampling approximation}

The likelihood we are going to use is based on the image formation
model
\begin{equation}
  y_{nm} = a\,\normal(\xi_m\given\mu_n,V_n) + \sigma\,g_{nm}
  \quad,
\end{equation}
where $y_{nm}$ is the $m$th pixel of image $y_n$,
$a$ is the overall amplitude of the 3-d Gaussian blob,
$\xi_m$ is the 2-d vector location of pixel $m$,
$\mu_n$ is the 2-d shift or center of the projected Gaussian blob,
$V_n$ is the projected 2-d variance tensor,
$\sigma$ is the amplitude of the iid noise,
and $g_{nm}$ is a draw from a univariate Gaussian of zero mean and unit variance.
(Apologies for overloading the $\sigma$ variable.
Alternate suggestions welcome.)

\end{document}
